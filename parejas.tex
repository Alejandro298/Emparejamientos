% Created 2019-11-07 jue 18:48
% Intended LaTeX compiler: pdflatex
\documentclass[presentation]{beamer}
\usepackage[utf8]{inputenc}
\usepackage[T1]{fontenc}
\usepackage{graphicx}
\usepackage{grffile}
\usepackage{longtable}
\usepackage{wrapfig}
\usepackage{rotating}
\usepackage[normalem]{ulem}
\usepackage{amsmath}
\usepackage{textcomp}
\usepackage{amssymb}
\usepackage{capt-of}
\usepackage{hyperref}
\usetheme{default}
\author{Invitado}
\date{\today}
\title{Emparejamientos}
\hypersetup{
 pdfauthor={Invitado},
 pdftitle={Emparejamientos},
 pdfkeywords={},
 pdfsubject={},
 pdfcreator={Emacs 25.2.2 (Org mode 9.2.3)}, 
 pdflang={English}}
\begin{document}

\maketitle
\begin{frame}{Outline}
\tableofcontents
\end{frame}


\begin{frame}[label={sec:org61e1c88}]{Definicion}
En una gráfica \(G\), un \alert{emparejamiento} es una coleccion de aristas ajenas.
Decimos que el emparejamiento \(M\) es \alert{perfecto} si todo
vertice de \(G\) está en alguna arista de \(M\).

Sea \(G=(A,B)\) una gráfica bipartita. Dado \(S\subseteq A\),
denotamos con \(N(S)\) al conjunto \(\{y\in B\mid\text{ existe }x\in
A\text{ con }y\sim x\}\)
\end{frame}

\begin{frame}[label={sec:org7f20916}]{Teorema de Hall}
Sea \(G=(A,B)\) una gráfica bipartita con \(|A|=|B|\). Entonces existe
un emparejamiento perfecto en \(G\) si y solo si para todo
\(S\subseteq A\) se tiene que \(|N(s)|\geq |S|\) 
\end{frame}
\end{document}